\cvprsection{AI 系统中的偏见类型与成因}

AI 系统的偏见可能在数据采集、模型训练、部署使用等多个阶段产生,主要偏见类型包括:

\begin{itemize}
	\item \textbf{表示偏见(Representation Bias)}:训练数据中不同群体的样本分布不均,少数群体样本不足,导致模型在这些群体上的性能下降。
	\item \textbf{历史偏见(Historical Bias)}:数据反映了现实社会中的已有歧视结构,导致模型继承甚至放大这些偏见。
	\item \textbf{测量偏见(Measurement Bias)}:不同群体的特征采集方式存在系统性差异,影响数据的客观性和一致性。
	\item \textbf{归纳偏见(Inductive Bias)}:模型设计时引入的假设或偏好,影响学习过程和结果。
	\item \textbf{算法偏见(Algorithmic Bias)}:算法优化目标或约束条件引发的偏差,例如过度追求整体准确率忽视群体间差异。
\end{itemize}

例如,Amazon 的招聘 AI 系统曾因训练数据仅基于过去历史简历记录,导致模型倾向于选择男性候选人,这种历史偏见严重影响系统的公平性\cite{Mehrabi2021Survey}。

偏见的产生机制复杂且相互叠加,任何一个阶段的偏见都可能影响最终决策,甚至导致严重的社会不公。因此,理解偏见来源对于公平 AI 的设计至关重要。