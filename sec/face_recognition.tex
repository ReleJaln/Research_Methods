\cvprsection{人脸识别中的偏见与伦理}

人脸识别是偏见问题最突出和关注度最高的领域之一。多项研究表明,商业人脸识别系统对有色人种、女性和儿童的识别准确率显著低于白人男性\cite{Buolamwini2018GenderShades}。例如,MIT Media Lab 的研究显示,部分商业系统对黑人女性的错误识别率超过 30\%,而对白人男性则低于 1\%。

偏见产生的主要原因包括:

\begin{itemize}
	\item 训练数据集中白人男性图像占比过高,数据代表性不足;
	\item 评估标准和测试集未涵盖多样化群体,难以发现模型中的系统偏差;
	\item 部署机构缺乏对伦理问题的敏感性和责任意识。
\end{itemize}

此外,人脸识别还引发隐私侵犯、无感授权和政府监控等伦理困境,成为社会广泛争议的焦点。一些欧美城市已经出台了对公共场合人脸识别的禁令,试图平衡技术发展与人权保护。