\cvprsection{高风险领域中的伦理冲突}

在司法、招聘和金融信贷等高风险决策领域,人工智能系统的应用正逐步扩大,其决策结果直接影响个人的自由、就业机会和经济权益,因此也引发了广泛的伦理与社会关注。以美国的量刑预测工具 COMPAS为例,研究表明该系统对黑人被告的再犯风险评分普遍偏高,进而导致司法实践中对少数族裔的不公正判罚\cite{Angwin2016MachineBias},成为算法歧视的典型案例。

在招聘领域,许多企业引入的自动筛选系统由于训练数据反映了过去的招聘偏见,倾向于优先筛选男性候选人,并排斥非英语母语者,削弱了系统的包容性和多样性。

在此背景下,学术界和产业界纷纷呼吁将“可解释性”、“争议权”和“人类监督权”纳入 AI 系统的设计原则,以提升系统的透明度、公平性与可控性。“可解释性”有助于揭示模型决策背后的依据,使潜在偏见得以识别;“争议权”允许受影响个体对 AI结果提出质疑与复核;“人类监督权”则确保关键决策始终有人类最终把关。这些机制的引入,是构建可信赖的 AI 系统、保障社会正义的重要路径。