\cvprsection{偏见缓解方法与公平性评估}

针对偏见,学术界和工业界提出了多种缓解策略:

\begin{itemize}
	\item \textbf{数据层面}:通过重采样、数据增强和多样化数据采集策略提升数据代表性;
	\item \textbf{模型层面}:采用对抗性训练、加权损失函数、因果建模等技术,减少模型偏差;
	\item \textbf{输出层面}:后处理阶段调整预测结果,如分组重校准,平衡不同群体的误差率。
\end{itemize}

公平性评估指标主要包括:

\begin{itemize}
	\item \textbf{Statistical Parity}:不同群体获得正向结果的概率相等;
	\item \textbf{Equal Opportunity}:在真实正样本中,各群体的正确预测率相等;
	\item \textbf{Predictive Parity}:不同群体的预测准确率相等;
	\item \textbf{Individual Fairness}:相似个体应有相似的预测结果。
\end{itemize}

不同应用场景需要根据业务目标和社会期望选择合适的公平性定义,并在准确性与公平性之间进行权衡。