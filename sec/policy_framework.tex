\cvprsection{法律与政策框架的最新进展}

2024年,欧盟通过了《人工智能法案(EU AI Act)》,提出基于风险等级的监管机制,将人脸识别、情感识别、社会评分列为高风险甚至禁止行为\cite{EuropeanParliament2024AIAct}。

美国提出《算法问责法案(Algorithmic Accountability Act)》,要求企业披露算法使用情况和影响报告,推动算法透明。

中国则通过《个人信息保护法》和《算法推荐规定》,强调用户知情权和算法透明,致力于构建公平透明的 AI 生态。

这些政策标志着全球对 AI 伦理治理的高度重视,未来挑战在于如何平衡技术创新与监管需求。